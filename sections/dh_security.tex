\subsection{Diffie-Helmann Security}
\subsubsection{Computational-Diffie-Hellman-Problem}
The Computational-Diffie-Hellman-Problem relies on the mathmatical discrete logarithm problem, which is used in more than one encryption procotoll.\\

\textbf{Definition 1.} Let $G$ a finite cyclic group of the order p. If $p$ is a prime number and $g$ a primitive root mod p then for every $A \in \{1,2,...,p-1\}$ there is exactly one exponent $a \in \{0,1,2,...,p-2\}$ with $A \equiv g^a \mod{n}$ \cite{kryptographie.2016}\cite{cryptMadeSimple.2015}. \\

Which menas \glqq the exponent $a$ is called \textit{discrete logarithm} of $A$ to the basis of $g$ \grqq \cite{kryptographie.2016}. Currently there is no suitable algorithm for an efficient calculation for this mathmatical problem known. The cumulative distribution function of the discrete logarithm appears to be very random for a big prime group. This leads to the fact, that the discrete logarithm can not calculated in a sufficient time for attackers.\\

When an attacker gets the numbers \textit{p}, \textit{g}, \textit{A} and \textit{B} he does not know the discrete logarithm. He has to calculate the secret key. So in fact the Computational-Diffie-Hellman-Problem is to calculate the secret key $K = g^{ab} \mod{p}$ \cite{kryptographie.2016}\cite{cryptMadeSimple.2015}\cite{Netzwerkkommunikation}.

\comment{
\textbf{Example 1.}
Let there be a group $\Z/n\Z$ with a number \textit{n} and cyclic order \textit{n}. Then a generator is  $1+n\Z$\\
Generator: $1+n\mathbb{Z}$\\
Congruence classs: $g + n\mathbb{Z}$
\\ $A\in \{0,1,...,n-1\}$\\
$\rightarrow A \equiv a \mod{n} \iff A \equiv g a \mod{n}$\\
}

\subsubsection{Decisional-Diffie-Hellman-Problem}

\subsubsection{Primenumber p and Bitlength q}
As conclusion of the Computational- and Decisional-Diffie-Hellman-Problem it is now known that the logarithm modulo \textit{p} is not efficient. But only if the group $G$ is big enough, which means the primenumber p must be also really big. Another security improvement is to increase the bitlength of q. Currently a good amount would be around 3000 bit \cite{kryptographie.2016}. As in the table \ref{tab:pp} is shown the recommend bitlength for q in the future is 512 and a minimum size p of 15,424. These numbers are necessary due to the increasing calucation power of the hardware. The 512 bit length for different usages is nowadays common and standard, besides 256 bit.
\begin{table}[ht]
	\centering
	\begin{tabular}{lcr}
		\toprule
		Proctection till  & Minimum size p & Bitlength q\\
		\midrule
		2015 & 1248 & 160\\
		2020 & 1776 & 192\\
		2030 & 2432 & 224\\
		2040 & 3248 & 256\\
		Future & 15,424 & 512 \\
		\bottomrule
	\end{tabular}
	\caption{Primenumber p and Bitlength q length estimation \cite{kryptographie.2016}}
	\label{tab:pp}
\end{table}

\subsubsection{Man in the middle attack}
Another attack on the Diffie-Hellman is the Man-In-The-Middle-Attack. This attack is quite common and can be used in a lot of situaitons. The attack infiltrates the communication between two instances Alice (A) and Bob (B). This makes it possible to influence the communication in a bad way. The attacker Mallory (M), the instance between A and B can spoofe and simulate towards A to be B and the other way round (see figure \ref{fig:mitm}).\\

In case of the Diffie-Hellman protocoll it means that Alice sends $g^a \mod p$ to Mallory instead of Bob. Mallory sends now $g^m \mod p$ with the own parameter \textit{m} to Bob and simulates over Bob to be Alice. For Bob it appears that Alice sent the request and sends $g^b \mod p$ back. Now M sends $g^b \mod p$ back to Alice. The secret key pair for A and M are $K_A = g^{am} \mod p$ and $K_{AM} = g^{am} \mod p$. For M and B the pair looks like this: $K_{BM} = g^{bm} \mod p$ and $K_B = g^{bm} \mod p$.\\

Mallory can now decrypt and encrypt every message that A and B sending each other and listens the communication.

\begin{figure}[h]
	\center
	\includegraphics[width=0.8\textwidth]{images/mitm.png}
	\caption{Man in the middle diagram}
	\label{fig:mitm}
\end{figure}
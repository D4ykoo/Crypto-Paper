\subsection{Diffie-Helmann Security}
\subsubsection{Computational-Diffie-Hellman-Problem}
The Computational-Diffie-Hellman-Problem relies on the mathmatical discrete logarithm problem, which is used in more than one encryption procotoll.\\

\textbf{Definition 1.} Let $G$ a finite cyclic group of the order p. If $p$ is a prime number and $g$ a primitive root mod p then for every $A \in \{1,2,...,p-1\}$ there is exactly one exponent $a \in \{0,1,2,...,p-2\}$ with $A \equiv g^a \mod{n}$ \cite{kryptographie.2016}\cite{cryptMadeSimple.2015}. \\

Which menas \glqq the exponent $a$ is called \textit{discrete logarithm} of $A$ to the basis of $g$ \grqq \cite{kryptographie.2016}. Currently there is no suitable algorithm for an efficient calculation for this mathmatical problem known. The cumulative distribution function of the discrete logarithm appears to be very random for a big prime group. This leads to the fact, that the discrete logarithm can not calculated in a sufficient time for attackers.\\

When an attacker gets the numbers \textit{p}, \textit{g}, \textit{A} and \textit{B} he does not know the discrete logarithm. He has to calculate the secret key. So in fact the Computational-Diffie-Hellman-Problem is to calculate the secret key $K = g^{ab} \mod{p}$ \cite{kryptographie.2016}\cite{cryptMadeSimple.2015}\cite{Netzwerkkommunikation}.

\comment{
\textbf{Example 1.}
Let there be a group $\Z/n\Z$ with a number \textit{n} and cyclic order \textit{n}. Then a generator is  $1+n\Z$\\
Generator: $1+n\mathbb{Z}$\\
Congruence classs: $g + n\mathbb{Z}$
\\ $A\in \{0,1,...,n-1\}$\\
$\rightarrow A \equiv a \mod{n} \iff A \equiv g a \mod{n}$\\
}

\subsubsection{Decisional-Diffie-Hellman-Problem}

\subsubsection{Primenumber p and Bitlength q}

\subsubsection{Man in the middle attack}

\subsection{El Gamal problems}
\subsubsection{Protocol problems}
Even though the ElGamal-Protocol is considered as safe this is only true when following the exact protocol. As explaination for the statement a look at random number \textit{r} will show it. If \textit{r} is used on more than one message the messages will be unsecure. This is because the exponiation of the group has to be calculated just once. This takes also affect when seeing through the perspective of an attacker. Even though he does not know the number \textit{r} the attacker can recognize a specific pattern and conlcude to \textit{r} and break the encryption.
\subsubsection{Subgroup problems}
Another problem is the same as in the Diffie-Hellman-Protocoll with the subgroups \cite{kryptographie.2016}\cite{Netzwerkkommunikation}\cite{cryptMadeSimple.2015}. Let define that \textit{q} of $\mathbb{G}$ is a prime number, then the trivial group is the only subgroup of $\mathbb{G}$. Then for every \textit{n} in \textit{q}, there is a subgroup $\mathbb{G}$ with the order \textit{n}. To identify if it is a element of the subgroup just proof if $h^n = 1$ \cite{kryptographie.2016}\cite{Netzwerkkommunikation}\cite{cryptMadeSimple.2015}. Therefore attacks are possible. The solution is a enough big subgroup to resist the attack which means the prime number has to be really big.
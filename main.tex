\documentclass[a4paper,11pt, twoside]{article}


%%%%%%%%%%%%%% Predefined packages
\usepackage{times,amsmath,eurosym, amssymb, color, graphicx}
\usepackage{booktabs, cellspace}
\usepackage[english]{babel} 
%%%%%%%%%%%%%% Packages

% Links
\usepackage[
hidelinks,
colorlinks=true, 
linkcolor=blue,
]{hyperref}
\usepackage{cleveref}

% Literatur
\usepackage[babel, german=quotes]{csquotes}
\usepackage[
backend=biber,				% Ref-Sortierung
style=numeric-comp,			% Zitationsstil
block=ragged, 				% Flattersatz
sorting=none,
]{biblatex}
\addbibresource{reference/reflist.bib}


% Images
\usepackage{color}
\usepackage{graphicx}
\graphicspath{{images/}}


%%%%%%%%%%%%%% Useful definitions
\newcommand{\R}{\mathbb R}
\newcommand{\C}{\mathbb C}
\newcommand{\F}{\mathbb F}
\newcommand{\Z}{\mathbb Z}
\newcommand{\Q}{\mathbb Q}
\newcommand{\N}{\mathbb N}

\newcommand\veczwo[2]{\left[\begin{array}{r}#1\\#2\end{array}\right]}
\newcommand\vecdrei[3]{\left[\begin{array}{r}#1\\#2\\#3\end{array}\right]}
\newcommand\vecdreic[3]{\left[\begin{array}{c}#1\\#2\\#3\end{array}\right]}
\newcommand\vecvier[4]{\left[\begin{array}{r}#1\\#2\\#3\\#4\end{array}\right]}
\newcommand\vecsechs[6]{\left[\begin{array}{r}#1\\#2\\#3\\#4\\#5\\#6\end{array}\right]}

\newcommand\matzz[4]{\left[\begin{array}{rr}#1&#2\\#3&#4\end{array}\right]}
\newcommand\matdd[9]{\left[\begin{array}{rrr}#1&#2&#3\\#4&#5&#6\\#7&#8&#9\end{array}\right]}
\newcommand\matddc[9]{\left[\begin{array}{ccc}#1&#2&#3\\#4&#5&#6\\#7&#8&#9\end{array}\right]}
%%%%%%%%%%%%%%% 
\newtheorem{definition}{Definition}
\newcommand{\comment}[1]{}


%%%%%%%%%%%%% Page layout
\topmargin-20mm
\headheight10mm
\headsep10mm
\topskip0.1mm
\hoffset-15mm
\textwidth15.5cm
\textheight22cm
\parindent0em
%\markleft{\authors}
%\markright{\shorttitle} 
\pagenumbering{arabic}
%%%%%%%%%%%%%%%%%%

%%%%%%%%%%%% Includes
\includeonly{
	sections/dh_security,
	sections/eg_problems,
}
%%%%%%%%%%%%

\begin{document}
\thispagestyle{empty}

\hrule

$\vphantom{i}$ \\[-.25cm]

\section{Diffie-Hellman and El Gamal}  
$\vphantom{i}\hspace{.65cm}$  \textit{Dario Köllner ...}

$\vphantom{i}$ \\[-.25cm]

\hrule

%%%%%%%%%%%%%%%%%%%%%%%%% ADD YOUR CONTRIBUTION BELOW
%%%%%%%%%%%%%%%%%%%%%%%%% USE SUBSECTION AND SUBSUBSECTION ONLY
% TODO: Abstract is mentioned in the evaluation criterias
\subsection*{Abstract}
\subsection*{Introduction}
todo all together...

%% DH part
\subsection{Diffie-Helmann Security}
\subsubsection{Computational-Diffie-Hellman-Problem}
The Computational-Diffie-Hellman-Problem relies on the mathmatical discrete logarithm problem, which is used in more than one encryption procotoll.\\

\textbf{Definition 1.} Let $G$ a finite cyclic group of the order p. If $p$ is a prime number and $g$ a primitive root mod p then for every $A \in \{1,2,...,p-1\}$ there is exactly one exponent $a \in \{0,1,2,...,p-2\}$ with $A \equiv g^a \mod{n}$ \cite{kryptographie.2016}\cite{cryptMadeSimple.2015}. \\

Which menas \glqq the exponent $a$ is called \textit{discrete logarithm} of $A$ to the basis of $g$ \grqq \cite{kryptographie.2016}. Currently there is no suitable algorithm for an efficient calculation for this mathmatical problem known. The cumulative distribution function of the discrete logarithm appears to be very random for a big prime group. This leads to the fact, that the discrete logarithm can not calculated in a sufficient time for attackers.\\

When an attacker gets the numbers \textit{p}, \textit{g}, \textit{A} and \textit{B} he does not know the discrete logarithm. He has to calculate the secret key. So in fact the Computational-Diffie-Hellman-Problem is to calculate the secret key $K = g^{ab} \mod{p}$ \cite{kryptographie.2016}\cite{cryptMadeSimple.2015}\cite{Netzwerkkommunikation}. \glqq As long as the calculation is hard enough that it can't be solved in a decent time, it is not possible to determine the secret key \grqq \cite{kryptographie.2016}.
\comment{
\textbf{Example 1.}
Let there be a group $\Z/n\Z$ with a number \textit{n} and cyclic order \textit{n}. Then a generator is  $1+n\Z$\\
Generator: $1+n\mathbb{Z}$\\
Congruence classs: $g + n\mathbb{Z}$
\\ $A\in \{0,1,...,n-1\}$\\
$\rightarrow A \equiv a \mod{n} \iff A \equiv g a \mod{n}$\\
}

\subsubsection{Decisional-Diffie-Hellman-Problem}
If the attacker can not get any information about the key, the Decisional-Diffie-Hellman-Problem (DDH) must be unassailable \cite{Boneh.1998}. At first the DDH seems to be harder than the Computational-Diffie-Helmlman-Problem but that is not the case \cite{kryptographie.2016} \cite{cryptMadeSimple.2015}. The attacker gets the following numbers $A = g^a \mod p, B = g^b \mod p, C = g^c \mod p$ \cite{kryptographie.2016}. Often times the Diffie-Hellman-Triple is set which means $c = ab \mod{p-1}$.\\
In other words $a, b, c$ are not known and $c$ is \grqq equal to [$a * b$] with probability $\frac{1}{2}$, otherwise it is [...] random \grqq \cite{cryptMadeSimple.2015}. In fact the attacker can not conclude to $g^{ab}$ and has to guess which case the challanger chose. The differenciation if there is triple can be solved with the following  context that \grqq $g^{ab}$ is a quadratic remainder modulo \textit{p}, if $g^a$ or $g^b$ a quadratic remainder modulo \textit{p} \grqq \cite{kryptographie.2016}. It can be solved two similiar ways. Let A computes $g^{a*b}$ the process to determine \cite{kryptographie.2016} \cite{cryptMadeSimple.2015} if wether it is or not, looks like the following \pagebreak

\indent $d \leftarrow A(a, b)$ \cite{cryptMadeSimple.2015} 
which achives the same result as \\
\indent $d \leftarrow ((\frac{A}{p}) = 1$ or $(\frac{B}{p}) = 1 )$ and $(\frac{C}{p}) = 1)$ \cite{kryptographie.2016}\\
\indent  \textbf{if} $d = c$  return \textit{1}\\
\indent \textbf{else} return \textit{0}\\

That proves that if the advantage of the Decisional-Diffie-Hellman-Problem over the Computational-Diffie-Hellman-Problem is wether equal \cite{cryptMadeSimple.2015} or even better \cite{kryptographie.2016}. In fact the CDH is harder to solve than the DDH \cite{kryptographie.2016} \cite{cryptMadeSimple.2015}.

\subsubsection{Primenumber p and Bitlength q}
As conclusion of the Computational- and Decisional-Diffie-Hellman-Problem it is now known that the logarithm modulo \textit{p} is not efficient. But only if the group $G$ is big enough, which means the primenumber p must be also really big. Another security improvement is to increase the bitlength of q. Currently a good amount would be around 3000 bit \cite{kryptographie.2016}. As in the table \ref{tab:pp} is shown the recommend bitlength for q in the future is 512 and a minimum size p of 15,424. These numbers are necessary due to the increasing calucation power of the hardware. The 512 bit length for different usages is nowadays common and standard, besides 256 bit.
\begin{table}[ht]
	\centering
	\begin{tabular}{lcr}
		\toprule
		Proctection till  & Minimum size p & Bitlength q\\
		\midrule
		2015 & 1248 & 160\\
		2020 & 1776 & 192\\
		2030 & 2432 & 224\\
		2040 & 3248 & 256\\
		Future & 15,424 & 512 \\
		\bottomrule
	\end{tabular}
	\caption{Primenumber p and Bitlength q length estimation \cite{kryptographie.2016}}
	\label{tab:pp}
\end{table}

\subsubsection{Man in the middle attack}
Another attack on the Diffie-Hellman is the Man-In-The-Middle-Attack \cite{Netzwerkkommunikation}\cite{angewandteKrypt.2020}. This attack is quite common and can be used in a lot of situaitons. The attack infiltrates the communication between two instances Alice (A) and Bob (B). This makes it possible to influence the communication in a bad way. The attacker Mallory (M), the instance between A and B can spoofe and simulate towards A to be B and the other way round (see figure \ref{fig:mitm}).\\

In case of the Diffie-Hellman protocoll it means that Alice sends $g^a \mod p$ to Mallory instead of Bob. Mallory sends now $g^m \mod p$ with the own parameter \textit{m} to Bob and simulates over Bob to be Alice. For Bob it appears that Alice sent the request and sends $g^b \mod p$ back. Now M sends $g^b \mod p$ back to Alice. The secret key pair for A and M are $K_A = g^{am} \mod p$ and $K_{AM} = g^{am} \mod p$. For M and B the pair looks like this: $K_{BM} = g^{bm} \mod p$ and $K_B = g^{bm} \mod p$.\\

Mallory can now decrypt and encrypt every message that A and B sending each other and listens the communication.

\begin{figure}[h]
	\center
	\includegraphics[width=0.8\textwidth]{images/mitm.png}
	\caption{Man in the middle diagram}
	\label{fig:mitm}
\end{figure}

$\vphantom{i}$ \\[-.25cm]

%% El Gamal part
\subsection{El Gamal problems}
\subsubsection{Protocol problems}
Even though the ElGamal-Protocol is considered as safe this is only true when following the exact protocol. As explaination for the statement a look at random number \textit{r} will show it. If \textit{r} is used on more than one message the messages will be unsecure. This is because the exponiation of the group has to be calculated just once. This takes also affect when seeing through the perspective of an attacker. Even though he does not know the number \textit{r} the attacker can recognize a specific pattern and conlcude to \textit{r} and break the encryption.
\subsubsection{Subgroup problems}
Another problem is the same as in the Diffie-Hellman-Protocoll with the subgroups \cite{kryptographie.2016}\cite{Netzwerkkommunikation}\cite{cryptMadeSimple.2015}. Let define that \textit{q} of $\mathbb{G}$ is a prime number, then the trivial group is the only subgroup of $\mathbb{G}$. Then for every \textit{n} in \textit{q}, there is a subgroup $\mathbb{G}$ with the order \textit{n}. To identify if it is a element of the subgroup just proof if $h^n = 1$ \cite{kryptographie.2016}\cite{Netzwerkkommunikation}\cite{cryptMadeSimple.2015}. Therefore attacks are possible. The solution is a enough big subgroup to resist the attack which means the prime number has to be really big.



%%%%%%%%%%%%%%%%%%%%%%%%% PUT IN A PRETTY PICTURE HERE
%\begin{figure}[h]
%\includegraphics[width=3.0cm, height=3.0cm]{path/filename}
%\end{figure}
%%%%%%%%%%%%%%%%%%%%%%%%% 

% TODO ALL OF US: Conclusion, critical reflection
\subsection{Conclusion}
The Diffie-Helmann can be considered as safe as long as the current mathmatical problems, discrete logoratihm and the guessing experiment of the DDH, can not be solved in a decent time and their are no fitting quantum computing algortihms to crack them. But instead of let the current implementations at a low bitlength and primenumber it should be always increased as time go by, to guarantee sustainable safety.\\

Due to the protocol Problems at El Gamal it should it be always implemented and used as it was original concipated, as well as using a good subgroup.
\printbibliography
\end{document}
